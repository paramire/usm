\documentclass[letter, 10pt]{article}
\usepackage[latin1]{inputenc}
%\usepackage[spanish]{babel}
\usepackage{amsfonts}
\usepackage{amsmath}
\usepackage[dvips]{graphicx}
\usepackage{url}
\usepackage[top=3cm,bottom=3cm,left=3.5cm,right=3.5cm,footskip=1.5cm,headheight=1.5cm,headsep=.5cm,textheight=3cm]{geometry}

\begin{document}
\title{Inteligencia Artificial \\ \begin{Large}Estado del Arte: Problema Stable Marriage\end{Large}}
\author{Patricio Ramirez Jofre}
\date{\today}
\maketitle


%--------------------No borrar esta secci\'on--------------------------------%
\section*{Evaluaci\'on}

\begin{tabular}{ll}
Resumen (5\%): & \underline{\hspace{2cm}} \\
Introducci\'on (5\%):  & \underline{\hspace{2cm}} \\
Definici\'on del Problema (10\%):  & \underline{\hspace{2cm}} \\
Estado del Arte (35\%):  & \underline{\hspace{2cm}} \\
Modelo Matem\'atico (20\%): &  \underline{\hspace{2cm}}\\
Conclusiones (20\%): &  \underline{\hspace{2cm}}\\
Bibliograf\'ia (5\%): & \underline{\hspace{2cm}}\\
 &  \\
\textbf{Nota Final (100\%)}:   & \underline{\hspace{2cm}}
\end{tabular}
%---------------------------------------------------------------------------%
\vspace{2cm}


\begin{abstract}
En este informe se representar\'a y explicar\'a el problema Stable Marriage, sus variaciones m\'as 
importantes, que tipo de aproximaciones existen para la resoluci\'on de estas, que situaciones modela o 
resuelve y qu\'e algoritmos son \'utiles. El Stable Marriage Problem (SM o SMP), consiste en un grupo de 
\textit{n} hombres y otro grupo de \textit{n} mujeres, los cuales cada uno de ellos tiene una lista de 
preferencia del sexo opuesto.	El problema consiste en crear matrimonios estables (Stable), o sea, que 
no exista deseo por otra persona o divorcio.	El problema SM puede resolverse en tiempo polinomial, 
pero las algunas variaciones de Stable Marriage with Incomplete lists and Ties (SMTI), pueden transformarse
en un problema NP-Hard. 
\end{abstract}

\section{Introducci\'on}

El prop\'osito de informe corresponde a la identificaci\'on, estudio y desarrollo del problema llamado Stable Marriage Problem, y sus variaciones 
existentes, tales como el ``Stable Marriage with Ties" (SMT), ``Stable Marriage with Incomplete Lists" (SMI) y una combinaci\'on de los anteriores
``Stable Marriage with Ties and Incomplete Lists" (SMTI), el tipo de problema mas ambicioso de resolver. 
Para ello se presentar\'an una o algunas definiciones del problema, explicando su funcionamiento y algunas representaciones para
las variaciones existentes, se ver\'an distintos algoritmos, t\'ecnicas, heur\'isticas y enfoques utilizados desde la formulaci\'on de este 
problema en 1962 por David Gale y Lloyd Shapley \cite{GaleShapley62}, ademas se presentaran los avances mas importantes, que cosas se ha hechos respecto
a este problema en la literatura y hacia donde avanza el estudio. \\

\indent El Stable Marriage Problem, consiste en \textit{n} hombres y \textit{n} mujeres, donde cada uno de ellos tiene una lista 
de preferencia ordenada con los miembros del sexo opuesto, siendo el objetivo crear parejas estables, o sea, que tanto hombre
como mujer no deseen divorciarse o separarse \cite{GaleShapley62}. Con el pasar de los a\~nos en las distintas variaciones
se ha refinado o modificado este problema agregando otras variables, conceptos y funciones que fortalecen el problema, se definen elementos como  
``Weakly Marriage" (Matrimonio Debil) y ``Strong Marriage" (Matrimonio Fuerte), y estos basados en ``Blocking Pairs" (Matrimonio donde Hombre
 y Mujer deseen estar con otra pareja).  \\ 
\indent SMP tiene muchas aplicaciones en la vida real, puede representar y solucionar distintas situaciones, como cierto tipo de problemas 
similares y muy relacionados con problemas de asignaci\'on, asignaciones Hospital-Residente, problemas de matching, etc. 
En consecuencia conocer las distintas de representaciones es de gran ayuda para resolver muchas situaciones del \'ambito 
de la Inform\'atica, Econom\'ia, Matematicas, etc,  siendo \'este un problema muy interesante de estudiar.\indent En el \'ambito de Inform\'atica,
 si bien las versiones SM, SMI y SMT pueden ser resueltas en tiempos polinomiales tanto SMTI y la escala de datos para hacer
 matching son desaf\'ios interesantes que pueden ser atacados por Algoritmos m\'as ambiciosos.\\ 
\indent	Este informe est\'a estructurado con tal de obtener un flujo de informaci\'on sobre el problema Stable Marriage Problem, comenzaremos definiendo el 
problema y sus variaciones m\'as conocidas, adem\'as de ciertas aproximaciones a ellas, comentare del Estado del Arte del problema, entregando ciertos 
conceptos y resultados obtenido para cada aproximaci\'on del problema, en la etapa final compartir\'e un modelo matem\'atico y finalizando con 
conclusiones respecto a la informaci\'on obtenida.    

\section{Definici\'on del Problema}

 El problema Stable Marriage, fue definido por David Gale y Lloyd Shapley en 1962 \cite{GaleShapley62},
 donde $m$ hombres y $n$ mujeres, crean listas de preferencias estrictamente ordenadas con todos los miembros del sexo opuesto, 
luego se define un \textit{Matching}, correspondiente a un par ordenado hombre-mujer, que define que 
el hombre \textbf{m} y la mujer \textbf{w} estan emparejados, ademas se definen los \textit{Blocking Pairs}, que ocurren para una pareja
$(m_i,w_i)$, el hombre $m_i$ prefiere a una mujer $w_j$ que a su actual pareja (o Matching), e la mujer $w_j$ prefiere a $m_i$ 
que a su pareja actual, de esto se define una pareja \textit{estable}, como aquellos Matching que no permiten un 
\textit{Blocking Pair} .
 Las variaciones mas conocidas son SM con listas incompletas, SM con Ties(Empates) y una combinaci\'on de los anteriores el 
SM with Incomplete Lists and Ties, 

\subsection{Stable Marriage with Incomplete Lists}
Para este problema, muy parecido al problema original, pero se permite que a cada uno de los hombres y mujeres, pueden tener
una lista incompleta de intergrantes del sexo opuesto. Es as\'i como en esta representaci\'on, podemos definir
que un hombre $m$ es \textit{aceptable} si aparece en la lista de una mujer $w$, e \textit{inaceptable}, en el de que no aparezca
en la lista de preferencias, esto se aplica contrariamente para las mujeres en las listas de preferencias de los hombres.
\indent Adem\'as cambia el concepto de \textit{estabilidad}, entonces un Matching M en una instancia I de SMI 
se considera estable si no existe un par (hombre, mujer), tal que para que cada uno:
\begin{itemize}
\item Se encuentren sin pareja en M y encuentran a otro \textit{aceptable} \'o
\item Prefieren a otra persona que a su pareja.
\end{itemize}

Esto implica para que un \textit{matching estable} no es necesario que sean \textit{matching} completo, esto quiere decir que puede existir un
grupo para hombres y mujeres que no se encuentran en pareja dentro del \textit{matching}.
Esto implica que todos los \textit{matching estables} para una instancia 
I de SMI, tienen el mismo tama\~no, o sea involucra exactamente la misma cantidad de hombres y mujeres \cite{ManloveIrving02Hard}

\subsection{Stable Marriage with Ties}

Otra variaci\'on importante de Stable Marriage Problem, surge cuando alg\'un hombre \'o mujer, requiere hacer una lista de 
preferencias estrictamente ordenada del sexo opuesto, tomando en cuenta que hay ciertos elementos del conjunto de preferencias
que son indiferentes entre si, es \'asi que se definen las \textit{ties} o empates, se define una instancia de SM, como SMT
cuando se admiten \textit{ties}, en consecuencia un \textit{Matching} se considera \textit{estable} cuando existe un par (hombre, 
mujer) que prefieren estar con otro que con su pareja, de estos criterios, se definen diferentes los conceptos de estabilidad:

\begin{itemize}
\item \textit{Weak stability} para un Matching M, no existe un par (m,w), donde cada uno de ellos prefiere estrictamente estar
con otro que con su pareja en M.
\item \textit{Strongly stability} para un matching M, cuando existe un par (m,w), donde el hombre $m$ prefiere 
estrictamente a la mujer $w$ y $w$ prefiere estrictamente a $m$.
\item \textit{Super Stability}, donde no existe un par (m,w) tal que alguno de ellos prefieran a estar con otra persona o es
 indeferente entre ellas.\cite{ManloveIrving02Hard,ThesisIoannis10}
\end{itemize} 
 

\subsection{Incomplete Lists with Ties}
	Para esta variaci\'on del problema debemos tener en cuenta las dos variaciones anteriores, ya que el problema \textit{Stable
Marriage Problem with Incomplete List and Ties}, genera una representaci\'on de este modelo combinando ambas ideas, las listas 
incompletas de SMI, y la indiferencia o \textit{ties} del SMT

\begin{table}[!h]
\centering
\begin{tabular}{l l}
$m_1: w_3 w_2$ & $w_1: m_1 m_2 $ \\
$m_2: (w_2,w_1) w_3$ & $w_2: m_1 (m_3 m_2) $ \\
$m_3: (w_2,w_1)$ & $w_3: ( m_1 m_2) $ \\
\end{tabular}
\caption{Ejemplo SMTI}
\end{table}

\indent Los conceptos de \textit{Weakly stable}, \textit{Strongly Stable} y \textit{Super Stable}, cambian
en base a las nuevas condiciones del problema, para un Matching M, se tienen \'asi los siguientes definiciones de estabilidad:

\begin{itemize}
\item \textit{Weakly stable} se define cuando, no existe un par (X,Y), donde cada uno de ellos se encuentra 
\textit{aceptable} o prefiere estrictamente estar con otro que con su pareja en M.

\item \textit{Strongly stable} sucede cuando un par (X,Y) tal que (a) \textbf{X} no tiene pareja en M y encuentra a \textbf{Y} aceptable o prefiere estrictamente a \textbf{Y} que su pareja en M,(b) \textbf{Y} no tiene pareja en M y encuentra a
\textbf{X} aceptable \'o prefiere estrictamente \textbf{Y} que a su pareja en M \'o \textbf{Y} es indiferente entre
\textbf{X} y su pareja en M .

\item \textit{Super stable} si no existe una par (x,y) tal que cada uno de ellos no tiene una pareja en M y encuentra al otro 
aceptable, \'o si prefiere estrictamente a su pareja en M \'o es indiferente entre ellos.\cite{ThesisPodhradsky10}
\end{itemize}

El problema SMTI es el mas ambicioso, ya que a diferencia de SMT, el largo entro los Matching posibles pueden ser de distintos tama\~nos, es asi que buscar un Matching Maximo(Un Matching mas largo)se convierte en un problema NP-Hard.
Es asi que algunos de los algoritmos que se hablaran en las siguientes secciones, buscan atacar la cardinalidad de los Matching
generados

AGREGAR COMPLEJIDADES

\subsection{Otras Variaciones}
\begin{itemize}
\item \textbf{Stable Marriage Problem}: en este problema no se definen dos grupos de personas, si no un solo gran grupo de $2n$ integrantes
donde cada uno de ellos crea una lista de preferencias para los $2n-1$ otros integrantes. Dadas estas condiciones este problema es 
muy similar a problema original de Stable Marriage Problem. \cite{IwamaMiyazaki08Survey}.
\item \textbf{Hospital/Resident Problem}: este problema representa asignaci\'on de pacientes a cada hosptial de la zona basado en su 
preferencia y la cuota del hospital. Se generan dos grupos, uno de residentes (R) y otro de hospitales (H) y para cada hospital una 
cuota asignada (C). Este problema se asemeja mas a el tipo SMI, considerando que un Hospital no se encuentra en el Matching si 
tiene menos residentes que la cuota.\indent Este problema se puede
reducir a un problema SM, definiendo que cada \textbf{C} instancias para un Hospital, una para cada posible Residente \cite{IwamaMiyazaki08Survey}.
\end{itemize}

\section{Estado del Arte} 

\subsection{SM}
El problema original \textit{Stable Marriage Problem}, se presento en el a\~no 1962, por David Gale y Lloyd Shapley en un 
articulo llamado \textit{College Admission and the Stability of Marriage} del journal \textit{American Mathematical Monthly} 
\cite{GaleShapley62} en esa publicaci\'on presentan ciertos algoritmos para encontrar un Matching estable en tiempo
$O(n^2)$, donde toma para cada hombre la mejor pareja posible, este caso es \textit{man-optimal} o sea que favorece al 
hombre esencialmente, aunque se aplique con tal de buscar la mejor pareja para las mujeres, este crearia un resultado 
\textit{woman-optimal}.

El algoritmo de Gale y Shapley siendo que encontraba matching estables pero poco favorables para alguno de grupos, ya sea los
hombres o las mujeres, es por eso que se buscan metodos para obtener matching estables basados en otros aspectos del problema
es por eso que que en 1989, Dan Gusfield and Robert W. Irving postulan metodos para encontrar matching estables\cite{Gusfield89},
mayoritariamente basados en los posici\'on de un hombre \textbf{m} o una mujer \textbf{w}, $p_w(m)$ corresponde a la posici\'on de
la mujer \textbf{w} para el hombre \textbf{m}, de esto se define

\begin{align}
r(M) = \underset{(m,w)\in M}{max} max (p_m(w),p_w(m)) \label{regret} \\
c(M) = \underset{(m,w) \in M}{\sum} p_m(w) + \underset{(m,w) \in M}{\sum} p_w(m) \label{eqal} \\
d(M) = \underset{(m,w) \in M}{\sum} p_m(w) - \underset{(m,w) \in M}{\sum} p_w(m) \label{sexeq}
\end{align}

Para encontrar un \textit{regret stable matching}, \textit{eqalitarian stable matching} y \textit{sex-equal stable matching} se debe encontrar 
los valore minimos para \textit{regret cost} \eqref{regret}, \textit{egalitarian cost} \eqref{eqal}, y \textit{sex-equalness} \eqref{sexeq} 
respectivamente.\cite{Gusfield89}

\subsection{SMI y SMT}

En 1994 Robert W. Irving muestra que para un problema SMT se puede encontrar un matching \textit{weakly stable} 
o \textit{Super stable} en tiempo $O(n^2)$, para aquellos algoritmos que busquen estos tipos de Matching
 o la no existencia en caso del \textit{Super Stability}, adem\'as presenta un algoritmo basado en un extension del 
 algoritmo de Gale y Shapley, que puede obtener una soluci\'on \textit{Strongly Stable} si existe una soluci\'on en un 
 tiempo de ejecuci\'on $O(n^2)$\cite{Irving94}
 
\subsection{SMTI}

En 2002 Manlove prueba que MAX SMTI (Maxima Cardinalidad para los Matching en una instancia I de SMTI)
es del tipo NP-HARD, incluso para algunos casos muy restringidos con Ties, adem\'as 
muestra que MAX SMTI es NP-completo, cuando las Ties solo existen en un lado (Hombres o Mujeres), \'o si solo hay
un Tie por lista y el largo de las tie es igual a 2, estos resultados se aplicando cuando se busca la Minima Cardinaliad de 
los Matching\cite{ManloveIrving02Hard}.

\subsection{Algoritmos}

La mayoria de los Algoritmos aca presentados son \'utiles para atacar el problema de MAX SMTI (Maxima Cardinalidad de una 
instancia I de SMTI), que como se mencion\'o algunos son problemas NP-Hard, los algoritmos mas ambiciosos y mas espec\'ificos seg\'un su
definici\'on
\subsubsection{RandBrk}
	Un \textit{Randomized algorithm}, propuesto por Halld\'orsson 
en el 2002, definiendo que para un caso, donde solo existen \textit{ties} en la lista de mujeres, donde existe al menos un 
\textit{tie} de un largo especifico de 2. \indent RandBrk recibe una instancia de un problema SMTI y para un tie de la lista de 
preferencias de m, se quiebra con igual probabilidad ( si existe una
lista de $w_1$ y $w_2$, ambas tienen una probabilidad de $0.5$ de
quedar primero), de este resultado se crea una nueva instacia del 
tipo SMI, y estas se pueden resolver en tiempo polinomial, con alg\'un
algoritmo para ello (i.e Gale-Shapley Algorithm)\cite{MagnusKazuo04Rand}.

\subsubsection{LocalSearch}

Este metodo fue propuesto en el 2010 por Mirco Gelain\cite{Gelain10}  y b\'asicamente menciona:

\begin{itemize}
\item Para una instancia I de SMTI, comenzamos con un Matching M generado de manera Random o por generado por alguna heuristica
\item En cada paso siguiente, nos movemos en el vecindario a un nuevo Matching M, el neighborhood de M ($N(M)$), corresponden a todos los 
Matchings M luego de remover un \textit{Blocking Pair}.
\item Luego de movernos en el Neighborhood y remover un \textit{Blockin Pair} (m,w) desde M, las parejas de \textbf{m} y 
\textbf{w} ($m'$,$w'$) quedan solteros.
\item Nos movemos a un Matching $M' \in N(M)$, tal que $f(M')\leq f(M'')$, siendo $f(M) = nbp(M) + ns(M)$, con $nbp(M)$ el n\'umero 
de blockings pair en M y $ns(M)$  el n\'umero de solteros en M.
\item Si M es estable, su neighborhood/Vecindario esta vacio y se debe aplicar un restart (Random).
\end{itemize}

Durante todo el algoritmo, se mantiene las mejoras encontradas, notamos que si no se ha encontrado un Matching M estable, se utiliza 
el que tiene menor valor de la funci\'on de evaluaci\'on $f(M)$, y en caso de no encontrar, es aquel con la menor cantidad de solteros.

Para evitar el estancamiento en un m\'inimo local, hacemos un movimiento random con una probabilidad \textit{p}, basada en un 
par\'ametro del algoritmo. El algoritmo finaliza cuando se crea un Matching perfecto (Matching estable sin solteros), o luego de una 
cantidad especifica de pasos.

\subsubsection{Algoritmo Genetico}  
	Una implementaci\'on de un algoritmo genetico para el problema SM consiste en una representaci\'on de los cromosomas como una lista
(o arreglo) de hombres o mujeres que se encuentran en pareja, por ejemplo, el arreglo [2 , 3, 1], corresponde a la mujer 2 con el hombre 1, la mujer 3 con el hombre 1, etc.
	Para esta implementaci\'on son necesarios el arreglo y las listas de preferencias, estas pueden ser creadas de manera aleatoria 
o con alguna heur\'istica de selecci\'on. A trav\'es de las iteraciones las listas de preferencias se mantienen constantes , s\'olo trabajando 
con el arreglo. 

Los algoritmos gen\'eticos buscan resultados a trav\'es de t\'ecnicas de cruzamiento y mutaci\'on
\begin{itemize}
\item El \textbf{cruzamiento} se crea tomando dos nodos padres, combinandolos de una manera especifica un nodo hijo, en
base a ellos. La manera mas com\'un es eligiendo una posici\'on especifica, quebrar los nodos padres en esa posici\'on y tomando 
las piezas restantes crear los nodos hijos uniendo las piezas complementarias.\indent De todas maneras este metodo no
es posible de utilizar en Stable Marriage Problem, ya que la mayor\'ia de los resultados ser\'ian inconsistentes, o valores repetidos para 
un Matching, de esto, en 1999 Aldershof y Carducci utilizan un metodo llamado \textit{cyclic crossover}\cite{Aldershof99Gen}.

\begin{itemize}
\item Cyclic crossover consiste que de cromosoma padre usara un valor $x_1$ en una posici\'on espec\'ifica y lo ponemos en la misma 
posici\'on en el nodo Hijo. 
\item Luego, vemos en la misma posici\'on de $x_1$ en el segundo 
cromosoma padre, tomamos el valor correspondiente (llamado gen opuesto) y lo ponemos en el nodo Hijo 1, en la posici\'on correspondiente segun el nodo padre.
\item Ahora se elige el valor $x_2$, pero esta vez seg\'un la posici\'on en el primer nodo padre, ahora se elige el gen opuesto (en el segundo padre) y se pone en el nodo Hijo 1, seg\'un la posici\'on que se encuentra en padre.
\item Se avanza con el mismo procedimiento hasta que encontremos un 
n\'umero que ya exista en el Nodo Hijo 1, Los valores utilizados se copian al Nodo Hijo 2 (Que hasta ahora esta vaci\'o)
\item finalmente los valores incompletos en ambos nodos hijos se rellenan con los valores del segundo nodo Padre 
\end{itemize}

\begin{table}[!h]
\centering
\begin{tabular}{|c|c|c|}
Step & Padres & Hijos \\ \hline
1 & Padre 1: [ 3, \textbf{2}, 5, 1, 4 ] & Child 1 : [ , 2, , , ] \\
  & Padre 2: [ 1, \textbf{3}, 4, 2, 5 ] & Child 2 : [] \\ \hline
2 & Padre 1: [ \textbf{3}, 2, 5, 1, 4 ] & Child 1 : [ 3, 2, , , ] \\
  & Padre 2: [ \textbf{1}, 3, 4, 2, 5 ] & Child 2 : [] \\ \hline
3 & Parent 1: [ 3, 2, 5, \textbf{1}, 4 ] & Child 1 : [ 3, 2, , 1, ] \\
  & Parent 2: [ 1, 3, 4, \textbf{2}, 5 ] & Child 2 : [] \\ \hline
4 & Parent 1: [ 3, 2, \textbf{5}, 1, \textbf{4} ] & Child 1 : [3,2, ,1,] \\ 
  & Parent 2: [ 1, 3, 4, 2, 5 ] & Child 2 : [ , , 5, , 4 ] \\ \hline
5 & Parent 1: [ 3, 2, 5, 1, 4 ] & \textbf{Child 1 : [3,2,4,1,5]} \\ 
  & Parent 2: [ 1, 3, 4, 2, 5 ] & \textbf{Child 2 : [1,3,5,2,4]} \\ \hline
\end{tabular}
\caption{Cyclic Crossover, desde la posici\'on 2\cite{ThesisIoannis10}}
\end{table}


\item El metodo de \textbf{mutaci\'on}, tiene las mismas dificultades que el cruzamiento. Para SM se generan dos funciones de 
mutaci\'on donde se elige una de ellas cada vez que se utilize la mutaci\'on. La primera funci\'on se hace un swap entre un cierto 
porcentaje de elementos del vector, ya que al igual que el poblema de cruzamiento cambiar solo un valor implicaria un valor duplicado, 
y asi mantener la matching aceptable. La otra funci\'on que es posible utilizar, es seleccionando todos pares inestables (blocking
pairs) y elegir un valor random entre ellos y cambiar ese elemento con otro valor random
\end{itemize}

\section{Modelo Matem\'atico}
Un modelo matematico, formulado para programaci\'on lineal:\cite{ThesisPodhradsky10}
\subsection{Variables}
\begin{itemize}
\item $X_{i,j}$ corresponde al par $m_i$, $w_j$ si esta un Matching estable.\\
\item A corresponde un grupo de pares aceptados
\item Espacio de Busqueda: $2^{(m*w)}$
\end{itemize}
\subsection{Funcion Objetivo}
\begin{equation*}
max \underset{i}{\sum} \underset{j}{\sum} x_{i,j}
\end{equation*}

\subsection{Restricciones}
\begin{align*}
\forall w \underset{i}{\sum} x_{i,w} \leq 1 \\
\forall m \underset{j}{\sum} x_{m,j} \leq 1 \\
\forall(m,w) \in A: \underset{j}{\sum} x_{m,j} + \underset{i}{\sum} x_{i,w} - x_{m,w} \geq 1 \\
\forall (m,w) \notin A = 0 \\
\forall (m,w) \in {0,1}
\end{align*}


\section{Descripci\'on del algoritmo}
C\'omo fue implementando, interesa la implementaci\'on m\'as que el algoritmo gen\'erico, es decir,
si se tiene que implementar SA, lo que se espera es que se explique en pseudo c\'odigo la estructura
general y en p\'arrafo explicativo cada parte como fue implementada para su caso particular, si
se utilizan operadores se debe explicar por que se utiliz\'o ese operador, si fuera el caso de una
t\'ecnica completa, si se utiliza recursi\'on o no, etc. En este punto no se espera que se incluya
c\'odigo, eso va aparte.

\section{Experimentos}
Se necesita saber como experimentaron, como definieron par\'ametros, como los fueron modificando, cuales 
problemas se trataron, instancias, por que ocuparon esos problemas.

\section{Resultados}
Que fue lo que se logr\'o con la experimentaci\'on, incluir tablas y par\'ametros, gr\'aficos si fuera
posible, lo m\'as explicativo posible.

\section{Conclusiones}
De acuerdo a la introducci\'on que se hizo, entregar afirmaciones RELEVANTES basadas en los experimentos
y sus resultados.


\section{Conclusiones}
%%Stable Marriage, es un problema sencillo de explicar y a pesar de ser un tema relativamente nuevo, pero que con el tiempo han aparecido autores que 
%%han contribuido con otras aproximaciones del mismo problema con representaciones mas complejas como Stable Marriage problem with Incomplete List and 
%%Ties, 
%%se han estudiado aspectos de la resoluci\'on del problema, ya sea creando matching mas igualitarios entre hombres y mujeres que en las representaciones 
%%originales de Gale y Shapley generaban soluciones \textit{man-optimal} o \textit{woman-optimal}, de las variaciones de SM, ya sea SM, SMI, SMT y SMTI 
%%entregan modelos a problemas reales, por ejemplo crear parejas o compa\~neros de cuarto en la facultad de una universidad puede ser resuelto a traves 
%%de alguna variaci\'on del problema Stable Marriage.\\
%%\indent Al igual que se han creado nuevas variaciones del problema, con el tiempo se han estudiado y creado distintos algoritmos para poder resolver 
%%estos problemas, siendo el mas ambicioso SMTI, ya que las variaciones Stable Marriage Problem with Ties o Stable Marriage with Incomplete List, se han
%%encontrado distintos algoritmos que resuelven obtienen un Matching M en tiempos polinomiales entre el orden de $O(n log(n))$ y $O(n^4)$, variando seg\'
%%un el tipo de 
%%Matching que se desea encontrar, por ejemplo encontrar un Matching \textit{man-optimal}, un Matching \textit{Weakly Stable} o un Matching \textit{sex-
%%equalitarian}, pueden representar Matching distintos basados en sus criterios en tiempos de ejecuci\'on distintos.\\
%%\indent De todo esto el problema mas ambicioso no es encontrar un Matching \textit{estable}, para un problema SMTI, si no encontrar todos los matching 
%%con la maxima cardinalidad posible (MAX-SMTI) e incluso con la menor cardinalidad posible (MIN SMTI), ya que ambos problemas son NP-Hard y no pueden ser
%%resueltos en tiempos polinomiales. Actualmente se han presentado una serie de algoritmos para obtener soluciones de SM, como algoritmos geneticos que 
%%explora y explota soluciones en base a una soluci\'on incial para pasar a soluciones mas optimas de SM, los algoritmos LocalSearch y RndBrk atacan el 
%%NP-Hardness del problema SMTI, donde para ciertas condiciones (i.e Ties en un solo lado (Hombre o Mujer) de largo 2 para el algoritmo RndBrk),
%%encontrar soluciones para estos problemas. 
\section{Bibliograf\'ia}
\bibliographystyle{plain}
\bibliography{Referencias}



\end{document} 