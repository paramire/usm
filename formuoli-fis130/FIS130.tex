\documentclass[10pt,twocolumn]{article}
\usepackage[utf8]{inputenc}
\usepackage{fancyvrb}
\usepackage{fancyhdr}
\usepackage{verbatim}
\usepackage{graphicx}
\usepackage{rotating}
\usepackage{listings}


%% FOR THIS PAPER
\usepackage{dblfloatfix}
\usepackage{fixltx2e}
\usepackage{mwe}
\usepackage{lipsum}
\usepackage{amsmath,bm}
\usepackage{enumitem}

\parskip 1mm
\setlength{\topmargin}{0pt}
\oddsidemargin  0.5cm
\evensidemargin 0.5cm
\textwidth      15.5cm
\textheight     21.0cm
\headsep        4 mm\usepackage{fixltx2e}
\parindent      0.5cm
\setlength{\itemsep}{2pt}

\pagestyle{fancyplain}

\lhead{FIS 130}
\rhead{\bf \it Formuoli}
\lfoot{}
\cfoot{\bf \thepage}
\rfoot{}
\renewcommand{\footrulewidth}{0.4pt}

\title{Formuli}
\date{\includegraphics[scale=0.4]{images/formuoli.jpg}}
\begin{document}
\maketitle
\section{Definiciones}
\begin{itemize}
\item $n$: Numero de Moles
\item $m$: Masa
\item $M, PM$: Masa Molecular
\item $V_{rms}^2$: Velocidad cuadratica media
\end{itemize}

\section{Temperatura y Calor}
\begin{itemize}[noitemsep,topsep=0pt]
\item Celsius a Farenheit 
\[T_f = \frac{9}{5}T_c + 32\]
\item Farenheit a Celsius
\[T_c = \frac{5}{9}(T_f -32)\]
\item Calor Especifico $(c)$ y Calor Especifico molar $(c_{mol})$
\[ c = \frac{Q}{m\Delta T}\]
\[ c_{mol} = \frac{Q}{n\Delta T}\]
\item Transferencia de calor
\[ Q = m\cdot c \cdot \left(T_f - T_i\right) \]
\[ dQ = m\cdot c\cdot dT\]
\[ dQ = n\cdot c_{mol}\cdot dT\]
\item Calor Latente
\begin{itemize}[noitemsep,topsep=0pt]
\item Fusión
\[ Q = mL_f\]
\item Vaporización
\[ Q = mL_v\]
\end{itemize}
\item Flujo de Calor
\[ H = \frac{dQ}{dt} = -kA\cdot \frac{dT}{dx} \]
\item Flujo Calor entre \textit{n} superficies
\[ H = - {\left(\frac{k}{L}\right)}_e \cdot A \cdot (T_i - T_s) \]
\[ \frac{1}{\left(\frac{k}{L}\right)_e} = \frac{1}{\frac{k_1}{L_1}} + \ldots + \frac{1}{\frac{k_n}{L_n}} \]
\item Primera Ley de la Termodinamica
\[\Delta U = Q - W \]
\begin{itemize}
\item $Q > 0$, El sistema absorbe Calor. 
\item $Q < 0$, El sistema libera Calor.
\item $W < 0$, El sistema se le ejerce fuerza.
\item $W > 0$, El sistema ejerce fuerza.
\end{itemize}

\end{itemize}
\section{Gases}
\begin{itemize}
\item Presión
\[ P = \frac{F_N}{A} \]
\item Ecuación de los Gases Ideales
\[ PV = nRT \]
\[ P = R'\rho T \]
\[ R' = \frac{R}{PM} \]
\[ P = \frac{\rho V_{rms}^2}{3} \]
\item Trabajo realizado por un Gas
\[ W = \int P dV \]
\item Energia interna de \textit{n} moles
\[ U = g_l \frac{1}{2} nRT\]
\item Energia Cinetica de \textit{n} moles
\[ E_k = g_l \frac{1}{2} nRT \]
\end{itemize}


\section{Unidades}
\begin{itemize}
\item Presion
\[\left[\frac{N}{m^2}\right] =  \left[Pa\right] \]
\end{itemize}
%\lipsum[1-4]
\end{document}